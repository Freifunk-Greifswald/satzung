\documentclass[parskip=half]{scrartcl}
\usepackage[utf8]{inputenc}
\usepackage[T1]{fontenc}
\usepackage[ngerman]{babel}
%\usepackage{lmodern}
\usepackage{scrjura}
\begin{document}
	\title{Satzung} 
	\author{Die Gründungssatzung des Freifunk Greifswald}
	\date{}
	%\tableofcontents
	%\addchap{Satzung}
	\maketitle
	\begin{contract}
		\Paragraph{title={Name, Sitz, Geschäftsjahr}}
		\begin{enumerate}
			\item Der Name des Vereins lautet \textit{Freifunk Greifswald}. Er soll in das Vereinsregister eingetragen werden. Nach der Eintragung führt er den Namenszusatz \textit{e.V.}
			\item Der Verein hat seinen Sitz in Greifswald.
			\item Das Geschäftsjahr des Vereins ist das Kalenderjahr.
		\end{enumerate}
		\Paragraph{title={Zweck des Vereins}}
		\begin{enumerate}
			\item Der Zweck des Vereins ist die Erforschung, Anwendung, Weiterentwicklung und Verbreitung freier Netzwerktechnologien sowie die Verbreitung und Vermittlung von Wissen zu Funk- und Netzwerktechnologien.
			\item Weiterhin fördert der Verein unmittelbar, ideell, materiell und/oder finanziell:
			\begin{itemize}
				\item den Aufbau eines frei zugänglichen Funknetzes (\textit{Freifunk}) im öffentlichen Raum
				\item den Zugang zu Informationstechnologien an öffentlichen Plätzen und Einrichtungen
				\item den Zugang zu Informationstechnologien für sozial benachteiligte Personen
				\item die Schaffung experimenteller Kommunikations- und Infrastrukturen sowie Bürgerdatennetze
				\item kulturelle, technologische und soziale Bildungs- und Forschungsprojekte
				\item die Veranstaltung regionaler, nationaler und internationaler Kongresse im Sinne des §2(1), Treffen und Konferenzen, sowie die Teilnahme der Mitglieder an diesen.
			\end{itemize}
		\end{enumerate}
		\Paragraph{title={Gemeinnützigkeit, Auflösung und Vermögen}}
		\begin{enumerate}
			\item Der Verein ist frei und unabhängig.
			\item Er verfolgt ausschließlich und unmittelbar gemeinnützige Zwecke im Sinne des Abschnitts \textit{Steuerbegünstigte Zwecke} der Abgabenordnung.
			\item Er ist selbstlos tätig und verfolgt nicht in erster Linie eigenwirtschaftliche Zwecke. Die Mittel des Vereins dürfen nur für die satzungsgemäßen Zwecke verwendet werden. Die Mitglieder erhalten keine Zuwendungen aus den Mitteln des Vereins.
			\item Es darf keine Person durch Ausgaben, die dem Zweck des Vereins fremd sind, oder durch unverhältnismäßig hohe Vergütungen begünstigt werden.
			\item Bei Auflösung der Körperschaft oder bei Wegfall steuerbegünstigter Zwecke fällt das Vermögen des Vereins an den \textit{Förderverein Freie Netzwerke e.V.}, welcher es unmittelbar für gemeinnützige Zwecke verwenden darf.
			\item Ausscheidende Mitglieder haben keinen Anspruch auf das Vereinsvermögen.
			\item Über die Auflösung des Vereins entscheidet eine Mitgliederversammlung, die eigens zu diesem Zweck einberufen wird. Die Auflösung gilt als beschlossen, wenn mindestens 75\% der abgegebenen Stimmen dafür stimmen.
		\end{enumerate}
		\Paragraph{title={Mitgliedschaft}}
		\begin{enumerate}
			\item Der Verein besteht aus ordentlichen Mitgliedern, Fördermitgliedern sowie aus Ehrenmitgliedern.
			\item Mitglieder können natürliche und juristische Personen, z.B. Firmen, Vereine, Verbände und Behörden werden, die gewillt sind, die gemeinnützigen Ziele des Vereins zu fördern und diesen in der Durchführung seiner Aufgaben zu unterstützen. Körperschaften, Vereine und Verbände können die Mitgliedschaft entweder nur für sich selbst oder auch für ihre Mitglieder erwerben. Bei Minderjährigen ist die Zustimmung des gesetzlichen Vertreters erforderlich.
			\begin{enumerate}
				\item Der Aufnahmeantrag ist schriftlich, auch in elektronischer Form, an den Vorstand zu richten, der über die Aufnahme des Antragstellers entscheidet.
				\item Eine Ablehnung des Aufnahmeantrags ist nicht anfechtbar und muss nicht begründet werden.
				\item Das aufgenommene Mitglied erhält eine Kopie der Satzung. Die jeweils aktuelle Satzung wird darüber hinaus an geeigneter Stelle den Mitgliedern verfügbar gemacht.
				\item Der Beitritt gilt erst dann als vollzogen, wenn der Mitgliedsbeitrag entrichtet worden ist.
				\item Mitglieder des Vereins sind verpflichtet eine gültige E-Mailadresse zur Zustellung von Vereinsinformationen anzugeben.
				\item Die Mitglieder haben das Recht, an der Mitgliederversammlung des Vereins teilzunehmen, Anträge zu stellen, und das Stimmrecht auszuüben. Juristische Personen üben ihr Stimmrecht durch bevollmächtigte Vertreter aus. Das aktive Stimmrecht besitzen Mitglieder mit Erreichen des 16. Lebensjahres. Das passive Wahlrecht beginnt mit Vollendung des 18. Lebensjahres.
				\item Jedes Mitglied hat einen Jahresbeitrag zu leisten, dessen Höhe und Fälligkeit in der Finanzordnung festgehalten ist. Diese wird von der Mitgliederversammlung beschlossen.
				\item Der Vorstand kann der Mitgliederversammlung die Ernennung von Ehrenmitgliedern vorschlagen. Ehrenmitglieder sind von Beitragszahlungen freigestellt und haben auf Mitgliederversammlungen volles Stimmrecht.
				\item Auf Antrag kann der Vorstand Mitgliedsbeiträge stunden und ganz oder teilweise erlassen.
				\item Die Mitgliedschaft endet durch Austritt, Ausschluss oder Tod bzw. bei juristischen Personen mit deren Erlöschen.
				\item Der Austritt muss durch schriftliche Mitteilung an den Vorstand erklärt werden. Er wird mit Ende des Geschäftsjahres wirksam und muss sechs Wochen vor dessen Ablauf mitgeteilt worden sein. Auf Wunsch des Mitglieds kann die Wirksamkeit auch mit sofortiger Wirkung eintreten.
				\item Der Ausschluss erfolgt durch den Vorstand und kann u. a. erfolgen
				\begin{itemize}
					\item bei schwerem Verstoß gegen die Vereinssatzung und bei anderem vereinsschädigenden Verhalten
					\item bei Rückstand in der Zahlung der Vereinsbeiträge von mehr als drei Monaten oder der Nichterfüllung sonstiger Mitgliedschaftlicher Verpflichtungen gegenüber dem Verein
					\item aufgrund von diskriminierenden Äußerungen oder Handlungen gegenüber den in Artikel 3 Absatz 3 des Grundgesetzes genannten Merkmalen
				\end{itemize}
				Das ausgeschlossene Mitglied kann innerhalb eines Monats nach Zugang des Beschlusses Einspruch einlegen und die nächste Mitgliederversammlung anrufen, von der die Gültigkeit des Ausschlusses bestätigt werden kann. Vom Zeitpunkt des Einspruchs bis zur Entscheidung über den Ausschluss ruht die Mitgliedschaft.
				\item Gezahlte Beiträge werden bei Beendigung der Mitgliedschaft nicht erstattet, auch nicht anteilig.
			\end{enumerate}
		\end{enumerate}
		\Paragraph{title={Die Organe des Vereins}}
		Die Organe des Vereins sind die Mitgliederversammlung und der Vorstand.
		\begin{enumerate}
			\item Die Mitgliederversammlung
			\begin{enumerate}
				\item Die ordentliche Mitgliederversammlung findet einmal jährlich statt.
				\item Der Vorstand hat eine außerordentliche Mitgliederversammlung unverzüglich und unter genauer Angabe von Gründen einzuberufen, wenn es das Interesse des Vereins erfordert oder mindestens 10\% der Mitglieder dies schriftlich unter Angabe des Zwecks und der Gründe vom Vorstand verlangen.
				\item Die Mitgliederversammlung ist bei ordnungsgemäßer Einladung ohne Rücksicht auf die Anzahl der erschienenen Mitglieder beschlussfähig. Sie wählt aus ihrer Mitte einen Versammlungsleiter oder eine Versammlungsleiterin. Beschlüsse werden, sofern die Versammlung nicht etwas anderes bestimmt, mit einfacher Mehrheit getroffen.
				\item Die Beschlüsse der Mitgliederversammlung werden in einem Protokoll niedergelegt und mit den Unterschriften des Versammlungsleiters oder der Versammlungsleiterin und des Protokollführers oder der Protokollführerin beurkundet.
				\item Der Mitgliederversammlung obliegt:
				\begin{enumerate}
					\item Beschlussfassung über alle den Verein betreffenden Angelegenheiten von grundsätzlicher Bedeutung
					\item Entscheidung über fristgemäß eingebrachte Anträge
					\item Entgegennahme des Jahresberichtes des Vorstands
					\item Entlastung des Vorstands
					\item Wahl der Vorstandsmitglieder
					\item die Genehmigung des vom Vorstand aufgestellten Haushaltsplans für das nächste Geschäftsjahr
					\item Beschlussfassung über Satzungsänderungen
					\item eine Änderung des Zweckes des Vereins oder der diesbezüglichen Satzungsbestimmungen ist lediglich unter Beachtung der Vorschriften gemäß §2, Gemeinnützigkeit, möglich
					\item Beschluss der Finanzordnung
					\item die Auflösung des Vereins gemäß §3, Ziffer 5 und 7 dieser Satzung
				\end{enumerate}
				\item Fristen:
				\begin{enumerate}
					\item Die Versammlung wird mindestens vier Wochen vor dem Versammlungstermin mit einer Mitteilung per E-Mail an die Mitglieder angekündigt. Das Einladungsschreiben gilt dem Mitglied als zugegangen, wenn es an die letzte vom Mitglied des Vereins schriftlich bekannt gegebene E-Mailadresse gerichtet ist.
					\item Ein Antrag an die Mitgliederversammlung gilt als fristgemäß eingereicht, wenn er eine Woche vor Beginn der Mitgliederversammlung beim Vorstand eingegangen ist.
				\end{enumerate}
			\end{enumerate}
			\item Der Vorstand
			\begin{enumerate}
				\item Der Vorstand des Vereins besteht aus drei Personen: Der oder die 1. Vorsitzende, der oder die 2. Vorsitzenden und der Schatzmeister oder die Schatzmeisterin sind Vorstand im Sinne des § 26 des Bürgerlichen Gesetzbuches. Jeder von Ihnen vertritt allein den Verein gerichtlich und außergerichtlich.
				\item Die Vorstandsmitglieder werden von der Mitgliederversammlung auf die Dauer von jeweils zwei Jahren gewählt. Sie bleiben bis zur Wahl des nächsten Vorstandes kommissarisch im Amt.
				\item Scheidet ein Vorstandsmitglied während der Amtszeit aus, so haben die übrigen Vorstandsmitglieder eine Ergänzung herbeizuführen, die der Bestätigung durch die nächste Mitgliederversammlung bedarf.
				\item Die Vorstandsmitglieder üben ihr Amt ehrenamtlich aus.
				\item Dem Vorstand obliegt die laufende Geschäftsführung, die Ausführung der Beschlüsse der Mitgliederversammlung und die Verwaltung des Vereinsvermögens.
				\item Der Vorstand kann zur Unterstützung und Wahrnehmung seiner Aufgaben Vereinsmitglieder berufen, die entweder auf Dauer oder nur zur Erfüllung einer zeitlich begrenzten Tätigkeit Funktionen übernehmen.
				\item Der Vorstand tagt mindestens einmal halbjährlich. Jedes Mitglied hat das Recht, an den Sitzungen des Vorstands teilzunehmen. Die Ergebnisse der Sitzungen sind zu dokumentieren und zeitnah zu veröffentlichen.
			\end{enumerate}
		\end{enumerate}
		\Paragraph{title={Schlussbestimmung}}
		Der Vorstand ist befugt, redaktionelle Änderungen an dieser Satzung, sofern sie einer Auflage des Registergerichtes oder einer Behörde entsprechen müssen, durchzuführen.
	\end{contract}
	\bigskip
	\hrule\bigskip
	Diese Satzung wurde auf der Gründungsversammlung am 08.12.2015 in Greifswald beschlossen.\vspace{2cm}
\end{document}
